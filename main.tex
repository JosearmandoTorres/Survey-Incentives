\documentclass[12pt]{article}

% ============================================
% PACKAGES
% ============================================

% Page layout
\usepackage[margin=1in]{geometry}

% Font and encoding
\usepackage[T1]{fontenc}
\usepackage[utf8]{inputenc}
\usepackage{indentfirst}

% Math
\usepackage{amsmath}
\usepackage{amssymb}

% Graphics and figures
\usepackage{graphicx}
\usepackage{float}
\usepackage{subcaption}

% Tables
\usepackage{booktabs}
\usepackage{tabularx}
\usepackage{multirow}
\usepackage{array}

% Citations and bibliography
\usepackage{natbib}
\bibliographystyle{apa} 

% Hyperlinks (load last among most packages)
\usepackage{hyperref}
\hypersetup{
    colorlinks=true,
    linkcolor=black,
    citecolor=black,
    urlcolor=blue,
    breaklinks=true
}

% Line spacing
\usepackage{setspace}
\onehalfspacing

% For appendix
\usepackage{appendix}

% For landscape pages (if needed for wide tables)
\usepackage{pdflscape}

% For editorial comments and TODOs
\usepackage{xcolor}
\newcommand{\todo}[1]{\textcolor{red}{[TODO: #1]}}
\newcommand{\plan}[1]{\textcolor{blue}{[PLAN: #1]}}
\newcommand{\addcite}[1]{\textcolor{orange}{[ADD CITATION: #1]}}
\newcommand{\new}[1]{\textcolor{blue}{#1}}  % For marking new content added in revision

% ============================================
% CUSTOM COMMANDS
% ============================================

% For easier table notes
\newcommand{\tablenote}[1]{\par\smallskip\footnotesize #1}

% ============================================
% TITLE PAGE SETUP
% ============================================

\title{\Large\textbf{Incentivizing Participation:\\
Developing an Evidence-Based Guide for Best Practices in Survey Incentives}}

\author{
    Jon A. Krosnick\thanks{Stanford University. Email: Krosnick@stanford.edu}
    \and
    Resty Fufunan\thanks{Stanford Institute for Research in the Social Sciences. Email: resty@stanford.edu}
    \and
    Josearmando Torres\thanks{Stanford Institute for Research in the Social Sciences. Email: jtorres0@stanford.edu}
}

\date{\today}

% ============================================
% DOCUMENT BEGINS
% ============================================

\begin{document}

% Title
\maketitle

% ============================================
% ABSTRACT
% ============================================

\begin{abstract}
In an era defined by challenges to data integrity and declining public trust, the pursuit
of representative sample survey data has never been more critical. Falling response
rates are thought to threaten the validity of public opinion research and its role in
informed decision-making. While survey incentives are a common industry practice
for boosting response rates, their application is far from standardized. Researchers
face complex choices regarding an incentive's form (cash, voucher), timing (prepaid
vs. promised), and amount, with little clear guidance on how to maximize participation
while minimizing costs.\\

This project, which is being developed as a section of a larger manual on optimal
survey design, is part of the ongoing work of the newly-established Stanford Institute
for Excellence in Survey Research (SIESR). This review of survey incentives directly
supports SIESR's mission to advance survey methodology and disseminate findings
not only to academics but also to media and policymakers. Synthesizing findings from
a wide-ranging literature review of experimental research, this project moves beyond
academic study to propose a decision-making framework for practitioners, fostering
methodological innovation and helping to maximize confidence in survey data.\\

Our research synthesizes quantitative findings to provide evidence-based
recommendations structured around the key dimensions of incentive design:\\
\begin{enumerate}
    \item Timing and Form: Analyzing the powerful effect of prepaid (unconditional)
incentives, which consistently and significantly outperform promised
(conditional) payments. We also differentiate the efficacy of monetary versus
non-monetary incentives, with a particular focus on the nuances between cash
and cash alternatives (gift card, check, etc.)
    \item Amount and Diminishing Returns: Examining the non-linear relationship
between incentive amount and response uplift to identify optimal payment
levels, providing clear guidance on cost-benefit ratios and diminishing marginal
returns.
    \item Survey Mode: Addressing how incentive effectiveness is moderated by data
collection environments, particularly across mail, web, and mixed-mode
surveys.
\end{enumerate}
Crucially, this project connects directly to the conference's focus on building trust in
data examining survey incentives as they relate to two critical, practical concerns:\\
\begin{itemize}
    \item Data Quality: Assessing the finding that incentives generally do not adversely
affect data quality (e.g., no increase in item nonresponse, satisficing, or
measurement error).
    \item Sample Representativeness: Investigating the impact of incentives on sample
composition, with a focus on how incentives can effectively boost response
rates among historically hard-to-reach populations
\end{itemize}
By providing a rigorous, data-driven framework for incentive design, this manual
serves as an innovative tool to help researchers improve methodological
transparency, maximize representativeness, and ultimately, reaffirm the critical role of
high-quality survey research.

\end{abstract}

\thispagestyle{empty}  % No page number on title page
\newpage
\setcounter{page}{1}



\end{document}